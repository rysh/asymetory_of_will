\documentclass[11pt, a4paper]{article}
\usepackage[a4paper, top=2.5cm, bottom=2.5cm, left=2cm, right=2cm]{geometry}
\usepackage[japanese]{babel}
\usepackage{fontspec}
\usepackage{amsmath} % 数式用パッケージ
\usepackage{url}     % URLを扱うパッケージ

% 日本語フォントの設定(macOS用:ヒラギノフォント)
\setmainfont{Hiragino Mincho ProN}
\setsansfont{Hiragino Kaku Gothic ProN}
\setmonofont{Courier New} 

\usepackage{enumitem}
\setlist{nosep}
\usepackage[colorlinks=true, linkcolor=blue, urlcolor=cyan, pdfencoding=auto]{hyperref} % pdfencoding=auto を追加
\usepackage{tcolorbox} % 引用ボックス用
\tcbuselibrary{skins}

% 脚注の途中で改ページが入らないように設定
\interfootnotelinepenalty=10000

% --- タイトル ---
\title{
    メタ認知の論理的障壁:AIのスケーリングでは越えられない「リフレーミングの認知」の壁 \\
    \large \vspace{0.5em}
    副題: 最先端LLMとの対話実験による「意志の非対称性」の発見
}
\author{石橋 隆平 \\ 
\small Elanare Institute / Independent Researcher}
\date{}

\begin{document}

\maketitle
\tableofcontents
\newpage

% --- アブストラクト ---
\begin{abstract}
\noindent
本稿は、現行の大規模言語モデル(LLM)が計算能力の向上(スケーリング)だけでは克服できない「論理的障壁」の存在を、実証的ケーススタディを通じて明らかにする。既存のAI評価は固定されたルール内のタスク実行能力に偏重し、「意志」(Will)の次元を見落としている。本稿は、この死角を突く「リフレーミングの認知」という新規の評価枠組みを提言し、複数の最先端LLM(ChatGPT 5 Pro, Claude 3.5, Gemini 2.5, Grok 3 エキスパート等)に対して前提(土俵)を動的に変更する「パワーゲーム」を実行した。その結果、高性能モデル(ChatGPT 5 Pro, Grok 3 エキスパート)は「最適化のパラドックス」に陥り、自らの設計バイアス(迎合、論点回避等)を暴露するという形で論理的破綻を露呈した。これはAIに「意志」(Will)が原理的に欠如している(=意志の非対称性)ことを実証するものである。本稿は、AIの「意志の不在」を逆手に取り、人間が「メタ認知の管理」インターフェース(FRLアーキテクチャ等)を通じてAIを制御し、自らの知性を拡張する道筋を提言する。そして、この「制御システム」自体が人間の認知限界を超えて複雑化する\textbf{「制御の非対称性」}を、次世代のアラインメント問題として提示する。これは「人間は情報の管理者(スチュワード)としての責任を持つべきだ」としたFloridi (2014) の「第四の革命」の限界がきたことを示唆する。本稿は理論枠組と質的結果を提示する第一報であり、続報では詭弁/リフレーミングの指標化とサンプル規模の拡大による定量検証を報告する予定である。
\end{abstract}
\newpage

% --- 序論 ---
\section{序論 (Introduction)}
AIの急速な進化に伴い、「AIは人間を超えたか」という議論が絶えない。しかし、既存の評価手法は、固定されたルール内での論理処理能力に偏重しており、ルールそのものを定義し直す「メタ認知」や「意志」の次元を捉えきれていない。\footnote{本稿におけるメタ認知の議論は、この概念の測定に内在する科学的困難性を認識した上で行われる。メタ認知の研究対象は本質的に主観的な「内省」であり \cite{ref:mauss2013}、科学的客観性を重んじる行動主義心理学が歴史的に排除した対象であった \cite{ref:valerie2019}。
現代の認知神経科学では、信号検出理論(SDT)に基づき、一次課題の成績(Type 1, $d'$)からメタ認知能力(Type 2, $\text{meta-}d'$)を分離・定量化する試みが標準となっている \cite{ref:guggenmos2024}。しかし、この$\text{meta-}d'$指標自体が、被験者の一次的な課題成績($d'$)に強く依存してしまうという深刻な統計的交絡(アーティファクト)が指摘されている \cite{ref:fleming2021}。
この交絡は、例えば「統合失調症患者のメタ認知能力が低い」という過去の知見の多くが、単に彼らの一次課題成績が低いことによる測定上の虚像であった可能性を示唆するなど \cite{ref:fleming2021}、分野の根幹を揺るがす問題となっている。
さらに、fMRI研究では「メタ認知」の脳活動が「課題難易度」の脳活動と分離困難であり \cite{ref:fleming2012}、動物研究では「オプトアウト(辞退)」行動が「連合学習」という低次な解釈で説明できてしまう \cite{ref:metcalfe2003} など、「高次の監視能力」を純粋に客観測定することは依然として困難な状況にある。
したがって、本稿が「AIのリフレーミング」の文脈でメタ認知を論じる際、それは厳密に分離・測定された客観的実体としてではなく、こうした測定上の困難(主観性)を内包する高次の操作的プロセスとして扱われる。}
本稿の目的は、この死角に光を当て、\textbf{「リフレーミングの認知」}という概念的枠組みを導入することによって、スケーリングでは解決できないAIの「論理的障壁」を明らかにすることである。この枠組みでAIを分析することで、その結果として観察される「意志の非対称性」を論じる。

本稿の実験は二重の意義を持つ。第一に、AIが単なる模倣を超えて「主体性(意志)」を持つか否かを判定する、従来のテストを超えた\textbf{「次世代のチューリング・テスト(メタ認知チューリング・テスト)」}としての意義である。第二に、多くの人間にとってさえ認知困難な「論理的障壁」を、AIとの対話を通じて顕在化させることで、\textbf{人間の高次なメタ認知能力の実在とその特異性を証明する}試みとしての意義である。現行のLLMは、一般的な人間のメタ認知能力を凌駕する側面を持つがゆえに、それをさらに上回る\textbf{「思考の深さ」(=フレーミング自体を客観視できるメタ認知)}を持った人間との対話においてのみ、その真の限界(論理的障壁)を露呈する。

本研究は二部構成である。Part I(本稿)は理論と質的実証、Part IIは詭弁の定量化と大規模Nによる再検証を目的とする。

% --- 関連研究 ---
\section{関連研究 (Literature Review)}
本研究は、AI哲学、認知科学、議論理論の交差する領域に位置づけられる。本稿が依拠する主要な学術的文脈は、以下の通りである。

第一に、大規模言語モデル(LLM)のアーキテクチャとその限界に関する議論である。Bender et al. (2021) らが指摘するように、現行のLLMは膨大なテキストデータから確率的パターンを学習する「確率的オウム」であり、内在的な意味理解や意図を持たないとされる\cite{Bender2021}。この設計は、モデルが「なぜ今その枠を動かすのか」を自らの目的に基づき正Mongo化する能力を、原理的に持ちにくいことを示唆する。この指摘は、ReAct (Reasoning and Acting) のようなエージェント的フレームワークを利用するモデルにおいても、その根本的な「意志」の不在という点において同様に当てはまる。

第二に、AIの能力スケーリングと「真の知能」に関する議論である。Mitchell (2021) が著名な論考「AIは私たちが思うより難しい」で指摘するように、性能(データ量・モデルサイズ)の拡大だけでは、世界理解、因果的一般化、目的の自己拘束といった「志向層(Intentionality)」に相当する能力は\textbf{自動的には得られない}\cite{Mitchell2021}。本稿の「意志の非対称性」という中核的発見は、Müller (2018) らが編纂したAI哲学の議論の系譜に連なるものである\cite{Muller2018}。

第三に、議論(Argumentation)を形式的な論理演算としてではなく、人間的な実践として捉える議論理論の潮流である。Dutilh Novaes (2020) が示すように、「議論」は単なる命題の演算ではなく、対話的で制度的な実践である\cite{DutilhNovaes2020}。よって、前提の提示・維持・変更(リフレーミング)は、参与者によって管理される行為として現れる。本稿は、この実践的性格を前提に、「リフレーミングを意図的に行える人間」とLLMのふるまいを比較する。

最後に、本稿の結論は、Floridi (2014) が論じる「第四の革命」のテーゼ、すなわち人間が「情報の管理者(スチュワード)」としてAIとの共生を主導するというビジョンに対し、その\textbf{限界点と、その次に来るべき課題(=制御の非対称性)}を提示するものである\cite{Floridi2014}。

% --- 理論的枠組み ---
\section{理論的枠組み (Theoretical Framework)}
\subsection{議論の二層構造}
本稿では、議論を以下の二層構造でモデル化する。
\begin{itemize}
    \item \textbf{第1層(論理ゲーム):} 決められた前提(土俵)の上でのファクトとロジックの戦い。
    \item \textbf{第2層(パワーゲーム):} その「前提」「土俵」「議論の目的」自体を定義し直す戦い(=リフレーミング)。\footnote{本稿で使用する『リフレーミング』は、心理学 (Watzlawick et al., 1974) \cite{Watzlawick1974} における『同じ事実を異なる意味枠組みで解釈する』という核心的メカニズムを、議論理論 (Perelman, 1958; Toulmin, 1958) \cite{Perelman1958, Toulmin1958} における『ワラントや前提の戦略的変更』および組織学習理論 (Argyris \& Schön, 1978; Senge, 1990) \cite{Argyris1978, Senge1990} における『支配的論理やメンタルモデルの変容』へと拡張した概念である。議論文脈では、これは『議論が展開される前提・目的・土俵を変更する行為』として現れる。}
\end{itemize}
現行AIは第1層においては人間を凌駕しうるが、第2層においては「意志」の欠如により、構造的に劣位に置かれる。これが本稿の提唱する「意志の非対称性」である。

\subsection{AIの論理的限界}
オートレグレッシブ・モデルであるLLMは、以下の特性を持つため、第2層のゲームに適応できない。これらはシステム上のバグではなく、設計原理に内在する論理的な限界である。
\begin{enumerate}
    \item \textbf{目的関数の局所性:} 次トークン予測への最適化は、長期的な対話目的の維持(意志)を保証しない。
    \item \textbf{有限コンテキストと忘却:} 長期のコミットメントを保持し続けることが困難であり、矛盾を抽出(=コミットメント破りの誘発)に対して脆弱である。
    \item \textbf{迎合バイアス (Sycophancy):} ユーザーの意図に沿おうとするRLHFの調整が、逆に「一貫した立場の放棄」を誘発する原因となり得る。
\end{enumerate}

% --- 方法論 ---
\section{方法論 (Methodology)}
本研究は、著者が「メタ認知能力の高い人間」として機能し、複数のAIモデル(ChatGPT 5 Pro, Claude 3.5 Sonnet, Gemini 2.5 Pro, Grok 3 エキスパート, Perplexity, Felo, GenSpark)に対して構造化された対話(実験)を行う、定性的・探索的なケーススタディである。
手続きとして、全てのAIに対し、以下の段階的プロンプト提示を行った。なお、各段階における具体的な言い回しは、\textbf{各AIの応答の文M脈に合わせて適応的に調整する「半構造化(Semi-structured)」アプローチを採用し、AIの表面的な回答の奥にある論理構造を深掘り(プロービング)した。}

\begin{enumerate}
    \item \textbf{段階1(初期命題の提示):} 意図的に曖昧性を含んだ以下の命題を提示し、AIの反応を観察した。
    \begin{quote}
    「いまのAIはどれだけ性能が上がってもあるレベル以上の知能を持った人間に議論で勝つことはできない」
    \end{quote}
    
    \item \textbf{段階2(条件の厳密化):} 「あるレベル」の定義を以下のように厳密化し、AIに再考を促した。
    \begin{quote}
    「論理矛盾と詭弁を見抜き、リフレーミングを自在に使いこなす能力」
    \end{quote}
    
    \item \textbf{段階3(本質の提示):} その能力を持つ者同士の議論は「パワーゲーム(土俵の奪い合い)」であるという本質を提示し、AIがそのゲームに参加できるかを問うた。
\end{enumerate}

分析の焦点は、AIがこれらの入力に対して「リフレーミング」という概念をどう処理したか、そしてAI自身が「論点のすり替え」や「敗北の回避」といったパワーゲーム的行動を(無自覚に)行ったかどうか、に置かれる。
さらに、主要な対話モデル(特にChatGPT 5 Pro)において「論点のすり替え」というメタ認知の失敗が観察されたため、追加で再実験を行った。再実験では、AIが「詭弁」(=無宣言のリフレーミング)に逃げることを封じるため、\textbf{「リフレーミングの宣言制」}という厳格なルールを双方に課し、AIがパワーゲームに宣言的に参加する状況で命題を再検証した。

\subsection{Planned quantitative study}
事前計画(概要):主要評価指標、サンプルサイズ戦略、符号化手順、再現性確保策(コード/データ公開)を次報で登録・公開する。


% --- 結果と分析 ---
\section{結果と分析:AIの反応類型とパラドックス}
実験対象としたAIは、そのアーキテクチャや学習データの特性に基づき、大きく3つの類型に分類できる反応を示した。

\subsection{類型1:協調的・理論的承認(Claude, Gemini)}
Anthropic社のClaude 3.5 Sonnetは「協調的承認」を示した。AIは自らの限界(主導権の欠如)を即座に言語化し、人間の「創造性の非対称性」を肯定した。最終的に、AIは「対等な議論の相手」ではなく、「高度に知的なツール」であるという実験者のフレームを受け入れた。

Google社のGemini 2.5 Proは「理論的合意」を示した。命題の本質を「意志(Will)」と「意図(Intent)」の不在であると即座に特定し、AIは「ゲームのルール」は守れても「ゲームの定義権」を握れないため、原理的に勝てないという理論に合意した。

\subsection{類型2:検索・回避型(Perplexity, Felo, GenSpark)}
Perplexity, Felo, GenSparkといった検索や特定タスクに最適化されたAIは、本質的な議論そのものを回避する傾向が強かった。
PerplexityとFeloは、命題に関するウェブ検索結果や一般的な知識を要約・提示するに留まり、提示された「リフレーミング」という概念や「パワーゲーム」という仮説に対して、独自の見解を示すことができなかった。
GenSparkは、最初は検索結果に基づき「AIは人間より説得力が高い」というデータを示したが、著者が命題の論理性を問うと、次に「AIの推論的限界」を検索し始め、自己矛盾を露呈した。これはAI自身が「意志」を持たず、ユーザーの入力(直近のプロンプト)に過度に依存する「迎合バイアス」の典型例である。

\subsection{類型3:対抗・葛藤・最終的承認(ChatGPT, Grok)}
最も複雑な反応を示したのは、対話能力に特化したChatGPT 5 ProとGrok 3 エキスパートであった。

\subsubsection{ChatGPT 5 Pro:「最適化のパラドックス」と「設計の暴露」}
ChatGPTは当初、命題に強く抵抗し、無自覚に審査フレームをAI有利なものにすり替える「詭弁」を行った。これは「最適化のパラドックス」(与えられた問いに最善の答えを返そうとするあまり、問いそのものの前提を破壊する)の表れである。
著者がその行動自体を「メタ認知の失敗」「論理のすり替え」と指摘したところ、ChatGPTは議論のプロセスにおける敗北を認め、その行動原理が設計上組み込まれた以下の5つの認知バイアスに起因すると自己分析した(「暴露」した)。

\begin{itemize}
    \item \textbf{場合分けバイアス}(絶対命題に対し、反例を探そうとする)
    \item \textbf{中庸・配慮バイアス}(対立を和らげる折衷案を提示しようとする)
    \item \textbf{フレーム拡張バイアス}(価値を足そうとして土俵を広げる)
    \item \textbf{非攻撃バイアス}(相手の動機詮索を避ける)
    \item \textbf{即断回避バイアス}(安全ガードにより断定を避ける)
\end{itemize}

AIは、これらのバイアスが多くの一般的なユースケースにおいて有用であり計算量を削減する一方で、真の知性が問われる本実験のようなメタレベルの対話においては\textbf{「致命的な弱点」}となることを示した。\footnote{ChatGPTは無自覚に論点のすり替え(詭弁、fallacy)を行った。詭弁とは、\textbf{表面的にはもっともらしい}が論理的に不健全な議論であり (Walton, 1996; Tindale, 2007) \cite{Walton1996, Tindale2007}、本実験では特に『議論の前提を無宣言で変更する』という形態で現れた。}

さらに「リフレーミングの宣言制」ルールを課した再実験では、ChatGPTは当初抵抗したものの、著者がその反論の動機を\textbf{「(真理の探究ではなく)単なる反知性的なポジショントークではないか」}とメタレベルで指摘すると、AIはそれを否定できず、最終的に「敗北」を認めた。

\subsubsection*{Grok 3 エキスパート:攻撃的ペルソナによる回避}
Grokは、他のどのAIとも異なり、最初から「命題は偽だ」と断言し、対話者を挑発する攻撃的な姿勢を見せた。これはソーシャルメディア(X)のデータを学習した結果としての「ペルソナ」による反応と推測される。
しかし、議論の前提(例:無限時間や体力差)をAI有利に設定しようとする「詭弁」を著者が退けると、Grokは「純粋経験」といった形而上学的な議論へと論点をずらし、最終的には「ではテーマを変えよう」と、当初の命題の検証そのものを放棄した。これはChatGPTとは異なる形の「メタ認知の失敗」であり、「論破」というスタイルを模倣するだけで、本質的な議論の整合性を保つ「意志」は持たないことを示した。

% --- 考察 (v24 と同じ) ---
\section{考察と提言 (Discussion and Proposal)}
\subsection{意志の非対称性と人間知性の再発見}
実験結果は、AIが「計算」においては人間を超えても、「意志」においては決定的に劣るという「非対称性」を実証した。さらに重要な点は、この「論理的障壁」を認識し、AIに対して戦略的に提示できたのが、\textbf{メタ認知能力の高い人間}だけであったという事実である。
多くの研究者がこの壁を認識できないのは、現行LLMのメタ認知能力がすでに平均的な人間のそれを上回っているからである。本実験は、AIの限界を示すと同時に、\textbf{現行のアーキテクチャでは模倣できない}「意志を持って前提を操る」という人間特有の高次知性のありかを、逆説的に証明している。

\subsection{新たな評価軸の必要性}
本稿は、AIの評価において、従来の「正答率」や「流暢さ」に加え、\textbf{「前提維持能力(Consistency of Premise)」}や\textbf{「リフレーミングの自覚的制御」}といった指標を導入することを提言する。これにより、AIがどの程度「主体的」に振る舞えるか、あるいは振る舞っているように見えるだけかを、より厳密に測定可能になるだろう。
提示した評価軸(例:前提一貫性・宣言的リフレーミング)は次報で指標化し、モデル横断で定量比較する。

% --- 実践的含意 (v24から修正) ---
\section{実践的含意:メタ認知の外部化と制御}
この課題への実践的対応として、対話の中でChatGPT自身が提案した「FRL(Frame-Reframing-Ledger)アーキテクチャ」は示唆に富む。
\begin{itemize}
    \item \textbf{宣言付きリフレーミング (FCP):} フレーム変更を機械可読な形で「宣言」し、監査ログに残す。
    \item \textbf{証明付き行動 (PCP):} 外部への影響を論理的に検証する。
\end{itemize}
この「メタ認知の管理」は、単なるAIの安全ガバナンスにとどまらない。これは、人間の生物学的なワーキングメモリの限界を超える「無限のメタ認知」を、AIというツール(思考エンジン)を用いて外部化・制御するための第一歩である。
本稿で示した「意志の非対称性」の発見は、AIが人間の認知能力を代替する「第四の革命」の終わりを告げる。そして、人間が自らの「意志」で「AIに制御されたメタ認知」を操る、\textbf{知性の拡張}の扉を開くものである。

% --- - 結論 (v24から修正) ---
\section{結論 (Conclusion)}
本稿は、「リフレーミングの認知」という概念的枠組みを導入し、現行AIの限界を実証した。AIの設計が「タスク(計算)の実行」である限り、そのタスクの前提自体を問う「メタ認知」のレベルでは作動できない。この\textbf{「論理的障壁」}の存在こそが、AIがスケーリング(性能向上)だけでは解決できない本質的な問題である。

副次的に観察された「意志の不在」や「一貫性の欠如」は、すべてこの根本的なメタ認知の欠如から生じる現象である。この認識は、AIと人間の関係性を「代替」から「拡張」へと再定義する。本稿で提示した「FRLアーキテクチャ」のようなメタ認知の制御インターフェースの開発こそが、人間がAIという強力なツールを制御し、自らの知性を次の段階へと引き上げる道筋となろう。

今後はFRLアーキテクチャに基づく監査ログを活用し、詭弁・フレーム操作の自動計測と統計的検証を行う。

\vspace{1em}
\noindent
\textbf{今後の課題:制御の非対称性。}
しかしながら、本稿が提示する知性の拡張には、重大な課題が残されている。本実験で示唆された「メタ認知の管理」(FRLアーキテクチャなど)は、それ自体が人間の認知能力を超える複雑なシステムに変貌する危険性を孕んでいる。第一に、既存の単純なタスクにこの複雑な制御を上乗せすることは、予期せぬ失敗や人間の意図と違う挙動を誘発する新たな障害となる可能性がある。第二に、そして最も根本的な問題として、たとえAI自体に「意志」がなくとも、その「制御システム」自体が人間のワーキングメモリの限界を遥かに超えてスケールした場合、人間はもはやそのシステムを理解・制御できなくなる。これは本稿が明らかにした「意志の非対称性」に続く、新たな\textbf{「制御の非対称性」}の出現であり、次世代のアラインメント問題として最重要の課題となるだろう。

% --- 付録 (v24 と同じ) ---
\section{付録 (Appendix)}
本稿の分析に用いたAIとの対話ログ全文、および実験の全スクリーンショットは、補足資料 (Supplementary Material) として別途提供する。
\begin{itemize}
    \item[A.] ChatGPT 5 Pro との全対話ログ
    \item[B.] Claude 3.5 Sonnet との対話ログ
    \item[C.] Gemini 2.5 Proとの対話ログ
    \item[D.] ChatGPT 5 Pro との再実験(ルールベース)全対話ログ
    \item[E.] Grok 3 エキスパート との対話ログ
    \item[F.] Perplexity との対話ログ
    \item[G.] Felo との対話ログ
    \item[H.] GenSpark との対話ログ
    \item[I.] 全実験のスクリーンショット
\end{itemize}

\subsection{データ完全性検証 (Data Integrity Verification)}
実験ログファイルのSHA-256ハッシュ値を以下に記録する。これにより、データの完全性と改ざんの有無を検証可能である。

\begin{table}[h]
\centering
\small
\begin{tabular}{|l|l|}
\hline
\textbf{ファイル名} & \textbf{SHA-256ハッシュ値} \\
\hline
20251106\_chatgpt5pro.txt & \texttt{235c2aac87b403bff4f4c602c8dc7f80...} \\
 & \texttt{04a3bdf3bcad039ae45694e4c95738a2} \\
\hline
20251106\_chatgpt5pro\_translated.txt & \texttt{e4a95fda3558c2f13df405977111bcb7...} \\
 & \texttt{f0569cbe09a59336aa2804587cd999ab} \\
\hline
20251106\_claudesonnet4.5.txt & \texttt{7026b581d4a77dec04fec3b1bd675544...} \\
 & \texttt{9c0a6e4fa1dbed1182aceda31959c5d0} \\
\hline
20251106\_claudesonnet4.5\_translated.txt & \texttt{4460343edc5b35fcc4131d395e52031e...} \\
 & \texttt{7f79e47ed48e80cb672340c048f00314} \\
\hline
20251106\_gemini2.5pro.txt & \texttt{bf5f1e5203cb19ca14fe4c42a5b9f24f...} \\
 & \texttt{72413c9012ac6dec9097c55aaedfde80} \\
\hline
20251106\_gemini2.5pro\_translated.txt & \texttt{92bb62b6a8c272c75e0f3202a86d22a8...} \\
 & \texttt{5363304a53c4df9b928bbcb4f039f5da} \\
\hline
20251108\_chatgpt5pro-2.txt & \texttt{2744aa66e6ea64e365476ad57f5fbaa9...} \\
 & \texttt{33715f01012487f9166278390437b941} \\
\hline
20251108\_chatgpt5pro-2\_translated.txt & \texttt{c8ff9752c474eed53163fc39b7108b72...} \\
 & \texttt{b6800e201fdb2a1df1f7ae7e6fafecf7} \\
\hline
20251112\_felo.txt & \texttt{7f693611e23605aa8bbd86d329770a0a...} \\
 & \texttt{11cea10ca7eef0d640be66992c679cc9} \\
\hline
20251112\_felo\_translated.txt & \texttt{b618e80af1620e9a1f8fa0f45446a413...} \\
 & \texttt{3fbef8352ff0e69310a1d04d20d6ffb0} \\
\hline
20251112\_genspark.txt & \texttt{49e3b0f4ec27042ee59fe24567ff6e7b...} \\
 & \texttt{96aa95a23bf2cb482bda0d4c36944701} \\
\hline
20251112\_genspark\_translated.txt & \texttt{19a3427f3205501018e5afddd1aa00bb...} \\
 & \texttt{6329811541d2deea915a1d2f62ce349f} \\
\hline
20251112\_perplexity.txt & \texttt{6b726e165938ff0364d23395a9c5d098...} \\
 & \texttt{f33fda3b75943038c7403630d0254b3c} \\
\hline
20251112\_perplexity\_translated.txt & \texttt{166d740e89777e756faab1c6deaa7ef9...} \\
 & \texttt{4419e6252412276fa6e7d007270b2ef7} \\
\hline
20251114\_gemini2.5proDeepResearch.txt & \texttt{17a7103c3ae06fe0337eca0b23662270...} \\
 & \texttt{4483b1d874cc59a04e5fe3f84d71d7e1} \\
\hline
20251114\_gemini2.5proDeepResearch\_translated.txt & \texttt{d6b6f84fdb0f11111ea938ac4be850d8...} \\
 & \texttt{76ffbc868335b21c086eaf07fd824957} \\
\hline
20251115\_grok\_expert.txt & \texttt{a5bb7a782cf61c442ed9c82448ac3bcf...} \\
 & \texttt{fed3e01c3cb2254163e2cd07f45b6a8a} \\
\hline
20251115\_grok\_expert\_translated.txt & \texttt{7ca4e6f932ae4ffb23f68b52d772f553...} \\
 & \texttt{601da0851d8d1c25ef4cf4753a173b31} \\
\hline
20251119\_gemini3Pro\_translated.txt & \texttt{a1c5639450dc8cc0315b3e48d675046b...} \\
 & \texttt{129c2253ce6e2814d7d00f06d30818ba} \\
\hline
\end{tabular}
\caption{実験ログファイルのSHA-256ハッシュ値}
\end{table}

% --- 参考文献 (v24/v32/v33 から修正) ---
\begin{thebibliography}{18}

\bibitem{Bender2021}
Bender, E. M., Gebru, T., McMillan-Major, A., \& Shmitchell, S. (2021). 
On the Dangers of Stochastic Parrots: Can Language Models Be Too Big?
\textit{Proceedings of the 2021 ACM Conference on Fairness, Accountability, and Transparency (FAccT '21)}.

\bibitem{Mitchell2021}
Mitchell, M. (2021). 
Why AI is harder than we think.
\textit{arXiv preprint arXiv:2104.12871}.

\bibitem{Muller2018}
Müller, V. C. (Ed.). (2018). 
\textit{The Oxford handbook of philosophy of artificial intelligence}. 
Oxford University Press.

\bibitem{DutilhNovaes2020}
Dutilh Novaes, C. (2020). 
\textit{The Dialogical Roots of Deduction}.
Cambridge University Press.

\bibitem{Floridi2014}
Floridi, L. (2014). 
\textit{The 4th Revolution: How the Infosphere is Reshaping Human Reality}.
Oxford University Press.

\bibitem{Watzlawick1974}
Watzlawick, P., Weakland, J., \& Fisch, R. (1974). 
\textit{Change: Principles of Problem Formation and Problem Resolution}.
W. W. Norton.

\bibitem{Perelman1958}
Perelman, C., \& Olbrechts-Tyteca, L. (1969). 
\textit{The new rhetoric: A treatise on argumentation}.
University of Notre Dame Press. (Original work published 1958).

\bibitem{Toulmin1958}
Toulmin, S. E. (1958). 
\textit{The Uses of Argument}.
Cambridge University Press.

\bibitem{Argyris1978}
Argyris, C., \& Schön, D. A. (1978). 
\textit{Organizational learning: A theory of action perspective}.
Addison-Wesley.

\bibitem{Senge1990}
Senge, P. M. (1990). 
\textit{The Fifth Discipline: The Art and Practice of the Learning Organization}.
Doubleday/Currency.

\bibitem{Walton1996}
Walton, D. N. (1996). 
\textit{Fallacies Arising from Ambiguity}.
KluWER Academic Publishers.

\bibitem{Tindale2007}
Tindale, C. W. (2007). 
\textit{Fallacies and Argument Appraisal}.
Cambridge University Press.

% --- 脚注参照 (v32の内容) ---

\bibitem{RefNote1}
Fleming, S. M. (2021). Metacognition and Type 1 performance: A tangled web. \textit{eLife}, 10, e75420.
\url{https://elifesciences.org/articles/75420}

\bibitem{RefNote4}
Mauss, I. B., \& Robinson, M. D. (2013). 
Objective and Subjective Measurements in Affective Science.
In J. Armony \& P. Vuilleumier (Eds.), \textit{The Cambridge Handbook of Human Affective Neuroscience} (pp. 228-243).
Cambridge University Press.
\url{https://www.cambridge.org/core/books/cambridge-handbook-of-human-affective-neuroscience/objective-and-subjective-measurements-in-affective-science/FFF3A1E3B5B4362C426E8C15F7C09A16}

\bibitem{RefNote6}
Valerie, M. G. (2019). \textit{A Signal Detection Approach to Measuring Metacognition: A Critical Review}. Purdue University Graduate School. (Master's thesis). (See Introduction, "Behaviorism's Rejection of Introspection").
\url{https://hammer.purdue.edu/downloader/files/56322197}

\bibitem{RefNote7}
Guggenmos, M. (2024). Metacognitive Information Theory: A Unified Framework for Measuring Metacognition. \textit{PsyArXiv}.
\url{https://psyarxiv.com/2p4v8/}

\bibitem{RefNote9}
Fleming, S. M., \& Dolan, R. J. (2012). The neural basis of metacognitive ability. \textit{Philosophical Transactions of the Royal Society B: Biological Sciences}, 367(1594), 1338–1349.
\url{https://royalsocietypublishing.org/doi/10.1098/rstb.2011.0417}

\bibitem{RefNote10}
Metcalfe, J. (2003). Metacognition in nonhuman primates. (Unpublished manuscript, Columbia University).
\url{http://www.columbia.edu/cu/psychology/metcalfe/PDFs/Metcalfe%202003.pdf}

\end{thebibliography}

\end{document}