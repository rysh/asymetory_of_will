% --- Preamble (v35 [115] と同じ) ---
\documentclass[11pt, a4paper]{article}
\usepackage[a4paper, top=2.5cm, bottom=2.5cm, left=2cm, right=2cm]{geometry}
\usepackage[japanese, bidi=basic, provide=*]{babel}
\usepackage{fontspec}
\usepackage{amsmath}
\babelprovide[import, onchar=ids fonts]{japanese}
\babelprovide[import, onchar=ids fonts]{english}
\babelfont{rm}{Noto Serif}
\babelfont[japanese]{rm}{Noto Serif CJK JP}
\usepackage{enumitem}
\setlist[itemize]{label=-}
\usepackage{hyperref}
\hypersetup{
    colorlinks=true,
    linkcolor=blue,
    filecolor=magenta,      
    urlcolor=cyan,
    pdftitle={メタ認知の論理的障壁},
    pdftitle={メタ認知の論理的障壁},
    pdfpagemode=FullScreen,
    }
\interfootnotelinepenalty=10000

% --- タイトル (v35 [115] と同じ) ---
\title{
    メタ認知の論理的障壁:AIのスケーリングでは越えられない「リフレーミングの認知」の壁 \\
    \large \vspace{0.5em}
    副題: 最先端LLMとの対話実験による「意志の非対称性」の発見
}
\author{石橋 隆平 \\ \small 独立研究者 (Independent Researcher)}
\date{}

\begin{document}

\maketitle
\tableofcontents
\newpage

% --- アブストラクト (v35 [115] と同じ) ---
\begin{abstract}
\noindent
本稿は、現行の大規模言語モデル(LLM)が計算能力の向上(スケーリング)だけでは克服できない「論理的障壁」の存在を、実証的ケーススタディを通じて明らかにする。既存のAI評価は固定されたルール内のタスク実行能力に偏重し、「意志(Will)」の次元を見落としている。本稿は、この死角を突く「リフレーミングの認知」という新規の評価枠組みを提言し、3つの最先端LLMに対して前提(土俵)を動的に変更する「パワーゲーム」を実行した。
その結果、高性能モデル(ChatGPT 5 Pro)は「最適化のパラドックス」に陥り、自らの「5つの設計バイアス」(迎合バイアス等)を暴露するという形で論理的破綻を露呈した。これはAIに「意志(Will)」が原理的に欠如している(=意志の非対称性)ことを実証するものである。
本稿は、AIの「意志の不在」を逆手に取り、人間が「メタ認知の管理」インターフェース(FRLアーキテクチャ等)を通じてAIを制御し、自らの知性を拡張する道筋を提言する。そして、この「制御システム」自体が人間の認知限界を超えて複雑化する\textbf{「制御の非対称性」}を、次世代のアラインメント問題として提示する。これは、「人間は情報の管理者(スチュワード)としての責任を持つべきだ」としたFloridi (2014) の「第四の革命」の限界がきたことを示唆する。
\end{abstract}
\newpage

% --- 序論 (v35 [115] と同じ) ---
\section{序論 (Introduction)}
AIの急速な進化に伴い、「AIは人間を超えたか」という議論が絶えない。しかし、既存の評価手法は、固定されたルール内での論理処理能力に偏重しており、ルールそのものを再定義する「メタ認知」や「意志」の次元を捉えきれていない。\footnote{本稿におけるメタ認知の議論は、この概念の測定に内在する科学的困難性を認識した上で行われる。メタ認知の研究対象は本質的に主観的な「内省」であり [4]、科学的客観性を重んじる行動主義心理学が歴史的に排除した対象であった [6]。
現代の認知神経科学では、信号検出理論(SDT)に基づき、一次課題の成績(Type 1, $\text{d'}$)からメタ認知能力(Type 2, $\text{meta-}d'$)を分離・定量化する試みが標準となっている [7]。しかし、この$\text{meta-}d'$指標自体が、被験者の一次的な課題成績($\text{d'}$)に強く依存してしまうという深刻な統計的交絡(アーティファクト)が指摘されている [1]。
この交絡は、例えば「統合失調症患者のメタ認知能力が低い」という過去の知見の多くが、単に彼らの一次課題成績が低いことによる測定上の虚像であった可能性を示唆するなど [1]、分野の根幹を揺るVASC問題となっている。
さらに、fMRI研究では「メタ認知」の脳活動が「課題難易度」の脳活動と分離困難であり [9]、動物研究では「オプトアウト(辞退)」行動が「連合学習」という低次な解釈で説明できてしまう [10] など、「高次の監視能力」を純粋に客観測定することは依然として困難な状況にある。
したがって、本稿が「AIのリフレーミング」の文脈でメタ認知を論じる際、それは厳密に分離・測定された客観的実体としてではなく、こうした測定上の困難(主観性)を内包する高次の操作的プロセスとして扱われる。}
本稿の目的は、この死角に光を当て、\textbf{「リフレーミングの認知」}という概念的枠組みを導入することによって、スケーリングでは解決できないAIの「論理的障壁」を明らかにすることである。この枠組みでAIを分析することで、その結果として観察される「意志の非対称性」を論じる。

本稿の実験は二重の意義を持つ。第一に、AIが単なる模倣を超えて「主体性(意志)」を持つか否かを判定する、従来のテストを超えた\textbf{「次世代のチューリング・テスト(メタ認知チューリング・テスト)」}としての意義である。第二に、多くの人間にとってさえ認知困難な「論理的障壁」を、AIとの対話を通じて顕在化させることで、\textbf{人間の高次なメタ認知能力の実在とその特異性を証明する}試みとしての意義である。現行のLLMは、一般的な人間のメタ認知能力を凌駕する側面を持つがゆえに、それをさらに上回る\textbf{「思考の深さ」(=フレーミング自体を客観視できるメタ認知)}を持った人間との対話においてのみ、その真の限界(論理的障壁)を露呈する。

% --- 関連研究 (v35 [115] と同じ) ---
\section{関連研究 (Literature Review)}
本研究は、AI哲学、認知科学、議論理論の交差する領域に位置づけられる。本稿が依拠する主要な学術的文脈は、以下の通りである。

第一に、大規模言語モデル(LLM)のアーキテクチャとその限界に関する議論である。Bender et al. (2021) らが指摘するように、現行のLLMは膨大なテキストデータから確率的パターンを学習する「確率的オウム」であり、内在的な意味理解や意図を持たないとされる\cite{Bender2021}。この設計は、モデルが「なぜ今その枠を動かすのか」を自らの目的に基づき正当化する能力を、原理的に持ちにくいことを示唆する。この指摘は、ReAct (Reasoning and Acting) のようなエージェント的フレームワークを利用する